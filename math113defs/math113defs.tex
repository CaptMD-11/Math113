\input{~/normal-preamble.tex}

\begin{center}
    Math 113 Definitions. 
\end{center}

\begin{enumerate}
    \item \textbf{Set. } A set is an unordered collection of elements. 
    \item \textbf{Map. } A map from $X$ to $Y$ is $f: X \to Y$ (a rule that assigns elements to $Y$ to elements in $X$). So, for any $x \in X$ there exists a unique $y \in Y$ such that $f(x)=y$. 
    \item \textbf{Cartesian Product. } The Cartesian product of $X$ and $Y$ is the set $X \times Y = \{(x,y) \mid x \in X, y \in Y\}$. 
    \item \textbf{Equivalence Relation. } An equivalence relation $R,\sim$ on $X$ is a subset $R \subseteq X \times X$ such that
    \begin{enumerate}
        \item Reflexive. ($(x,x) \in R$ for all $x \in X$). 
        \item Symmetric. (if $(x,y) \in R$, then $(y,x) \in R$). 
        \item Transitive. (if $(x,y) \in R$ and $(y,z) \in R$, then $(x,z) \in R$). 
    \end{enumerate}
    \item \textbf{Equivalence Class. } Let $X$ be a set and $R$ be an equivalence relation on $X$. Then, an equivalence class of $x \in  X$ is the set $[x]=[x]_R=[x]_{\sim} = \{a \in X \mid x \sim a\}$. 
    \item \textbf{$\mathbb{Z}/m\mathbb{Z}. $} The set of distinct equivalence classes of $\equiv \Mod{n}$ is $\mathbb{Z}/m\mathbb{Z}$.
	\item \textbf{Group. } A group $G$ (denote: $(G,\star)$) is a set $G$ with a closed binary operation $\star: G \times G \to G$ such that: 
	\begin{enumerate}
		\item Associativity: $(a \star b) \star c = a \star (b \star c)$ for all $a,b,c \in G$. 
		\item Identity: There exists an $e \in G$ such that for any $a \in G$, we have $a \star e = e \star a = a$. 
		\item Inverse: For any $a \in G$, there exists an $a^{-1} \in G$ such that $a \star a^{-1} = a^{-1} \star a = e$. 
	\end{enumerate}
	\item \textbf{Symmetric Group. } The symmetric group on $n$ letters is $S_n$. 
	\item \textbf{Disjoint Cycles. } Two cycles $(a_1,\dots,a_k)$ and $(b_1,\dots,b_l)$ are disjoint if $a_i \neq b_j$ for all $i,j$. 
	\item \textbf{Transpositions. } The simplest permutation is a cycle of length 2, which is called a transposition. 
	\item \textbf{Even, Odd Permuatations. } A permutation is even if it can be expressed as an even number of transpositions. A permutation is odd if it can be expressed as an odd number of transpositions. 
	\item \textbf{Subgroup. } A subgroup $H$ of a group $G$ is a subset $H$ of $G$ such that when the group operation of $G$ is restricted to $H$, then $H$ is a group. 
	\item \textbf{Trivial/Proper Subgroup. } The trivial subgroup of a group $G$ is $\{e\}$ and a proper subgroup is a subgroup $H$ of $G$ where $H$ is a proper subset of $G$. 
	\item \textbf{General/Special Linear Group. } $GL_2(\mathbb{R})$ is the set of 2x2 invertible matrices with real entries. $SL_2(\mathbb{R})$ is the set of 2x2 invertible matrices with real entries and with determinant 1.
	\item \textbf{Cyclic Group. } A cyclic group is a group generated by one element. 
	\item \textbf{Isomorphism. } An isomorphism is a homomorphism which is bijective. 
	\item \textbf{Kernel of homomorphism. } If $\phi: G \to H$ is a homomorphism, then $\ker \phi$ is the pre-image of $e_H \in H$, that is, $\ker \phi \{g \in G \mid \phi(g)=e_H\}$.
	\item \textbf{Coset. } Let $(G,\star) \geq (H,\star)$ and $g \in G$. Then, an $H-$ coset of $g$ is a (sub)set of $G$ where $gH = g \star H = \{g \star h \mid h \in H\}$ (left coset) and $Hg = \{h \star g \mid h \in H\}$ (right coset). 
	\item \textbf{Index. } A set of distinct equivalence classes with respect to $~_H$ is $G/H$, a quotient of $G$ by $H$. Then, $|G/H| = [G:H]$ is the index of $H$ in $G$. 
	\item The following are definitions listed in the homeworks: 
	\begin{enumerate}
		\item Group of units in $\mathbb{Z}/n\mathbb{Z}$ is the set $(\mathbb{Z}/n\mathbb{Z})^\times = \mathbb{Z}/n\mathbb{Z}^\times := \{[a] \in \mathbb{Z} \mid \exists [b] \in \mathbb{Z}/n\mathbb{Z} \textrm{ with } [a] \times [b] = [1]\}$. 
		\item If $G$ is a group, then the center of $G$ is the set $Z(G) := \{a \in G \mid ga=ag \forall g \in G\}$. 
		\item $\mathbb{C}^\times$ is the set of nonzero complex numbers. 
		\item $\mathbb{R}^\times$ is the set of nonzero real numbers. 
		\item $GL(n,K)$ is the set of $n$ x $n$ invertible matrices with entries in $K$. 
		\item If $G$ is a group, then the torsion subgroup of $G$ is called $G_T$, which is the set of all elements of $G$ with finite order.
		\item The Klein four-group is $V$ is a subgroup of $S_4$ and consists of $V=\{\id, (12), (34), (12)(34)\}$. 
	\end{enumerate} 
	\begin{center}
		\hrule
	\end{center}
	\item \textbf{Dihedral group. } This is the group of symmetries on a regular $n$-gon with $r$ being rotation $s$ flip. We have $r^n = \id$, $s^2 = \id$, and $srs = r^{-1}$. 
	\item \textbf{(External) Direct Product} Let $G = (G,\star)$ and $H = (H, \circ)$ be groups. Then, $G \times H = \{G \times H, (\star, \circ)\}$. 
	\item \textbf{Normal Subgroup. } Let $G$ be a group and $H$ a subgroup of $G$. Then $H$ is a normal subgroup (write $H \unlhd G$) iff for all $g \in G$, $gH=Hg$, or equivalently, for all $h \in H$, $ghg^{-1} \in H$ for all $g \in G$. 
	\item \textbf{Quotient (factor) group. } The quotient group of a group $G$ and a normal subgroup $N$ of $G$ is the group $G/N$ (where $G/N$ is the group of cosets of $N$ in $G$) under the operation $(aN)(bN) = abN$. 
	\item \textbf{Internal Direct Product. } Let $G$ be a group and $H,K \leq G$. $G$ is an internal direct product of $H$ and $K$ iff: 
	\begin{enumerate}
		\item $G=H \cdot K := \{h \cdot k \mid h \in H, k \in K\}$. 
		\item $H \cap K = \{e_G\}$ (\enquote{as small as possible}). 
		\item $h \cdot k = k \cdot h$ for all $h \in H, k \in K$. 
	\end{enumerate}
	\item \textbf{Simple group. } A group $G$ is simple if the only normal subgroups are $\{e_G\}$ are $G$. 
	\item \textbf{Symmetry. } A symmetry of $X$ is a bijective map $\sigma: X \to X$ preserving the structure where $X$ is some set with some additional structure. 
	\item \textbf{Group of permutations on a set $X$. } $G$ is a group of permutations on a set $X$ if $\phi: G \to \textrm{Sym}(X)=S_X=S_{|X|}$ is a homomorphism that is 1-1. 
	\item \textbf{$G$ acts on a set $X$. } $G$ acts on a set $X$ is a homomorphism $\phi: G \to \textrm{Sym}(X)$. 
	\item \textbf{Stabilizer of $x \in X$. } Let $G$ be a group and $X$ a set. Then, the stabilizer of $x \in X$ is $\Stab_G(x) = \{g \in G \mid g(x)=x\}$, which are elements of $g$ that preserve $x \in X$. 
	\item \textbf{Orbit of $x \in X$. } Take $x \in X$. Then the orbit of $X$ is $\orb_G(x) = \mathcal{O}_G(x) = \mathcal{O}(x) = \{g(x) \mid g \in G\} \subseteq X$. 
	\item \textbf{$G$ acts on a set $X$ (equivalent) def. } A group $G$ acts on a set $X$ iff $\Phi: G \times X \to X$ with $(g,x) \mapsto \Phi(g,x) = g \circ x$ such that: 
	\begin{enumerate}
		\item for all $x \in X$, $\Phi(e_G,x) = x$. 
		\item for all $x \in X$, $g,h \in G$, $\Phi(gh,x) = \Phi(g,\Phi(h,x))$. 
	\end{enumerate}
	\item \textbf{Left regular action of $G$ on $G$. } Define $\Lambda: G \times G \to G$ be a group action, where $G$ is a group (and a set) such that $(g,h) \mapsto g \circ h$. Equivalently, the left regular action of $G$ on $G$ is defined as the homomorphism $\lambda: G \to \Sym(G)$ such that $g \mapsto (\lambda_g: h \mapsto \lambda_g(h) = g \circ h)$, where $\lambda_g$ is a permutation on the set $G$ for $g \in G$. 
	\item \textbf{Ring. } A ring $R = (R,+,\times)$ is a set with two closed binary operations ($+$ and $\times$) such that: 
	\begin{enumerate}
		\item $(R,+)$ is an abelian group. 
		\item $(R,\times)$ is associative. 
		\item Both distrbutive properties hold, i.e. $a \times (b+c) = (a \times b) + (a \times c)$ and $(a+b) \times c = (a \times c) + (b \times c)$ for all $a,b,c \in R$. 
	\end{enumerate}
	\item \textbf{$R^\times$. } Let $R$ be a ring. Then, we define $R^\times = \{a \in R \mid \exists b \in R : ab = 1_R\}$, where $(R^\times, \times)$ is a group, possibly abelian. 
	\item \textbf{Field. } Let $R = (R,+,\times)$ be a commutative ring with $1_R$ being the multiplicative identity. Then if $R^\times = R \setminus \{0\}$, then $R$ is a field. 
	\item \textbf{Ring Homomorphism. } Let $R = (R,+,\times)$ and $S=(S,+,\times)$ be rings. Then a map $\phi: R \to S$ is a homomorphism of rings iff $\phi(a +_R b) = \phi(a) +_S \phi(b)$ and $\phi(a \times_R b) = \phi(a) \times_S \phi(b)$. Additionally, axiomatically, $\phi(1_R)=1_S$ and a property is $\phi(0_R) = 0_S$. Also, define $\ker\phi := \{r \in R \mid \phi(r) = 0_S\}$. 
	\item BELOW IS ADDITIONAL DEFS FOR MT2
	\item A \textbf{ring with unity (or with identity)} is a ring $R$ that has multiplicative identity. 
        \item A \textbf{commutative ring} is a ring $R$ that has multiplicative commutativity. 
        \item An \textbf{integral domain} is a commutative ring $R$ with identity such that for all $a,b \in R$ $ab=0$ implies $a=0$ or $b=0$. 
        \item A \textbf{division ring} is a ring $R$ that has multiplicative inverse for all nonzero $a \in R$. 
        \item A \textbf{zero divisor} of a commutative ring $R$ is an $a \in R$ ($a \neq 0$) such that there exists a nonzero $b \in R$ such that $ab=0$. 
        \item The \textbf{ring of quaternions} is the set $\mathbb{H} = \{a + b\hat{i} + c\hat{j} + d\hat{k} \mid a,b,c,d \in \mathbb{R}\}$, where $1 = \begin{pmatrix}
            1 & 0 \\
            0 & 1
        \end{pmatrix}, \hat{i} = \begin{pmatrix}
            0 & 1 \\
            -1 & 0
        \end{pmatrix}, \hat{j} = \begin{pmatrix}
            0 & i \\
            i & 0
        \end{pmatrix}, \hat{k} = \begin{pmatrix}
            i & 0 \\
            0 & -i
        \end{pmatrix}.$
	\item A \textbf{field} is a commutative division ring. 
        \item The \textbf{characteristic} of a ring R is the least positive integer $n$ such that $nr=0$ for all $r \in R$. If no such $n$ exists, the characteristic of $R$ is defined to be 0. (denote the characteristic of $R$ by $\textrm{char} R$). 
	\item A \textbf{ring homomorphism} is a map $\phi: R \to S$ (where $R,S$ are rings) such that $\phi(a+b) = \phi(a) + \phi(b)$ and $\phi(ab) = \phi(a)\phi(b)$ for all $a,b \in R$. 
        \item A \textbf{ring isomorphism} is a bijective map $\phi: R \to S$ where $R,S$ are rings. 
        \item The \textbf{kernel} of a ring homomorphism $\phi: R \to S$ is the set $\ker\phi := \{r \in R \mid \phi(r) = 0\}$. 
        \item An \textbf{evaluation homomorphism} is a ring homomorphism of the form $\phi_\alpha: C[a,b] \to \mathbb{R}$ or other such related homomorphisms. 
        \item An \textbf{ideal} of a ring $R$ is a subring $I$ such that 
	\begin{enumerate}
	\item $(I,+)$ is a subgroup of $(R,+)$. 
	\item if $a \in I$ and $r \in R$, then $ar,ra \in I$. 
	\end{enumerate}
        \item The \textbf{trivial ideals} of a ring $R$ are the subrings $\{0\}$ and $R$. 
        \item A \textbf{principal ideal} of a commutative ring $R$ (with identity) is an ideal of the form $\langle a \rangle = \{ar \mid r \in R\}$. 
        \item A \textbf{two-sided ideal} $I$ is a subring of a ring $R$ such that $rI \subset I$ and $Ir \subset I$ for all $r \in R$. 
        \item A \textbf{one-sided ideal} $I$ is a subring of a ring $R$ is one such that $rI \subset I$ for all $r \in R$ (a \textbf{left ideal}) or $Ir \subset I$ for all $r \in R$ (a \textbf{right ideal}). 
	\item \textbf{Quotient ring. } Let $R$ be a ring and $I$ a two-sided ideal of $R$. Then the quotient ring $R/I$ is defined to be the set of all cosets of $I$ with respect to $+$ and $\times$. 
	\item \textbf{Natural/canonical homomorphism. } The map $\phi: R \to R/I$ is called the natural/canonical homomorphism. 
	\begin{center}
		\hrule
	\end{center}
	\item \textbf{proper ideal. } $I \unlhd R$ is a proper ideal of $R$ iff $I \neq \{0_R\}$ and $I \neq R$. 
	\item \textbf{Integral domain. } A commutative ring $R$ with $1_R$ is an integral domain if there are no (nonzero) zero-divisors. 
	\item \textbf{Prime ideal. } An ideal $I$ of a ring $R$ is a prime ideal if $ab \in I$ means $a \in I$ or $b \in I$. 
	\item \textbf{Prime. } Let $p \in D$, where $D$ is an integral domain and $p$ a non-unit. $p$ is prime iff if $p \mid ab$, then $p \mid a$ or $p \mid b$. 
	\item \textbf{Irreducible. } Let $x \in D$, where $D$ is an integral domain and $x$ a non-unit. $x$ is irreducible iff if $x =ab$ means $a$ is a unit or $b$ is a unit. 
	\item \textbf{Principal ideal domain (PID). } A principal ideal is an integral domain in which every ideal is a principal ideal. 
	\item \textbf{Unique factorization domain (UFD). } An integral domain $D$ is a unique factorization doman (UFD) if: 
	\begin{enumerate}
		\item Let $a \in D$ such that $a \neq 0$ and $a$ is a non-unit. Then $a$ can be written as the product of irreducible elements of $D$. 
		\item Let $a = p_1 \cdot \cdot \cdot p_r = q_1 \cdot \cdot \cdot q_s$, where $p_i,q_k$ are irreducbile. Then $r=s$ and there is a $\pi \in S_r$ such that $p_i$ and $q_{\pi(j)}$ are associates for $j=1,\dots,r$. 
	\end{enumerate}
	\item \textbf{Euclidean domain. } Let $D$ be an integral domain such that there is a function $v: D \setminus \{0\} \to \mathbb{N}$ such that: 
	\begin{enumerate}
		\item If $a,b$ are nonzero elements of $D$, then $v(a) \leq v(ab)$. 
		\item Let $a,b \in D$ and suppose $b \neq 0$. Then There exist elements $q,r \in D$ such that $a=bq+r$ and either $r=0$ or $v(r) < v(b)$. 
	\end{enumerate}
	Then $D$ is a Euclidean domain. 
	\item \textbf{Gaussian Integers. } The set of Gaussian integers is the set $\{a + bi \mid a,b \in \mathbb{Z}, i^2 = -1\} =: \mathbb{Z}[i]$. 
	\item \textbf{Norm. } Let $z \in \mathbb{Z}[i]$. Then we define the norm of $z$ to be $N(z)=z \cdot \overline{z}$, or if $z=a+bi \in \mathbb{Z}[i]$, then $N(z) = a^2 + b^2$. 
	\item \textbf{Norm (again). } Norm of $z = a+b\sqrt{-5} \in \mathbb{Z}[\sqrt{-5}]$ is $a^2 + 5b^2 = z\overline{z} \in \mathbb{Z}$. 
	\item \textbf{Prime (again). } Let $p \in R$, non-unit. $p$ is prime iff if $p \mid ab$ then $p \mid a $ or $p \mid b$ for all $a,b \in R$. 
	\item \textbf{Irreducible (again). } $q \in R$ is irreducible if for any $a,b \in R$ such that $q = a \cdot b$, then $a$ or $b$ is a unit. 
\end{enumerate}


\end{document}
