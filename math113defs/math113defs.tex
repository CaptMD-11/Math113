\documentclass[12pt]{article}
% \usepackage[left=2cm, right=2cm, top=1.5cm, bottom=1.5cm]{geometry}
\usepackage{amsmath}
\usepackage{amsthm}
\usepackage{amsfonts}
\usepackage{amssymb}
\usepackage{authblk}
\usepackage{tkz-euclide}
\usepackage{tikz}
\usepackage{changepage}
\usepackage{lipsum}
\usepackage{tree-dvips}
\usepackage{qtree}
\usepackage[linguistics]{forest}
\usepackage[hidelinks]{hyperref}
\usepackage{mathtools}
\usepackage{blindtext}
% \usepackage[cal=esstix,frak=euler,scr=boondox,bb= pazo]{mathalfa}
% the following 2 packages are used for changing the font. 
\usepackage{mathptmx}
\usepackage{mathrsfs}
\usepackage{graphicx}
\usepackage{setspace}
\graphicspath{{./images/}}
\allowdisplaybreaks
\allowbreak
\theoremstyle{definition}
\newtheorem{definition}{Definition}
\newtheoremstyle{named}{}{}{\itshape}{}{\bfseries}{.}{.5em}{\thmnote{#3's }#1}
\theoremstyle{named}
\newtheorem*{namedconjecture}{Distinct Factorizations Conjecture}
\newtheorem{conjecture}{Conjecture}
\DeclareMathOperator{\sech}{sech}
\DeclareMathOperator{\arcsec}{arcsec}
\DeclareMathOperator{\lcm}{lcm}
\DeclareMathOperator{\curl}{curl}
\DeclareMathOperator{\Res}{Res}
\DeclareMathOperator{\Aut}{Aut}
\DeclareMathOperator{\id}{id}
\DeclareMathOperator{\nul}{nul}
\DeclareMathOperator{\End}{End}
\newcounter{customDef}
\renewcommand{\thecustomDef}{\arabic{customDef}}
\newcommand{\Mod}[1]{\ (\mathrm{mod}\ #1)}
\begin{document}

\begin{center}
    Math 113 Definitions. 
\end{center}

\begin{enumerate}
    \item \textbf{Set. } A set is an unordered collection of elements. 
    \item \textbf{Map. } A map from $X$ to $Y$ is $f: X \to Y$ (a rule that assigns elements to $Y$ to elements in $X$). So, for any $x \in X$ there exists a unique $y \in Y$ such that $f(x)=y$. 
    \item \textbf{Cartesian Product. } The Cartesian product of $X$ and $Y$ is the set $X \times Y = \{(x,y) \mid x \in X, y \in Y\}$. 
    \item \textbf{Equivalence Relation. } An equivalence relation $R,\sim$ on $X$ is a subset $R \subseteq X \times X$ such that
    \begin{enumerate}
        \item Reflexive. ($(x,x) \in R$ for all $x \in X$). 
        \item Symmetric. (if $(x,y) \in R$, then $(y,x) \in R$). 
        \item Transitive. (if $(x,y) \in R$ and $(y,z) \in R$, then $(x,z) \in R$). 
    \end{enumerate}
    \item \textbf{Equivalence Class. } Let $X$ be a set and $R$ be an equivalence relation on $X$. Then, an equivalence class of $x\in  X$ is the set $[x]=[x]_R=[x]_{\sim} = \{a \in X \mid x \sim a\}$. 
    \item \textbf{$\mathbb{Z}/m\mathbb{Z}. $} The set of distinct equivalence classes of $\equiv \Mod{n}$ is $\mathbb{Z}/m\mathbb{Z}$.
	\item \textbf{Group. } A group $G$ (denote: $(G,\star)$) is a set $G$ with a closed binary operation $\star: G \times G \to G$ such that: 
	\begin{enumerate}
		\item Associativity: $(a \star b) \star c = a \star (b \star c)$ for all $a,b,c \in G$. 
		\item Identity: There exists an $e \in G$ such that for any $a \in G$, we have $a \star e = e \star a = a$. 
		\item Inverse: For any $a \in G$, there exists an $a^{-1} \in G$ such that $a \star a^{-1} = a^{-1} \star a = e$. 
	\end{enumerate} 
\end{enumerate}

\end{document}
