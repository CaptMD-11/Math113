\input{~/normal-preamble.tex}

\begin{center}
    Math 113 Definitions. 
\end{center}

\begin{enumerate}
    \item \textbf{Set. } A set is an unordered collection of elements. 
    \item \textbf{Map. } A map from $X$ to $Y$ is $f: X \to Y$ (a rule that assigns elements to $Y$ to elements in $X$). So, for any $x \in X$ there exists a unique $y \in Y$ such that $f(x)=y$. 
    \item \textbf{Cartesian Product. } The Cartesian product of $X$ and $Y$ is the set $X \times Y = \{(x,y) \mid x \in X, y \in Y\}$. 
    \item \textbf{Equivalence Relation. } An equivalence relation $R,\sim$ on $X$ is a subset $R \subseteq X \times X$ such that
    \begin{enumerate}
        \item Reflexive. ($(x,x) \in R$ for all $x \in X$). 
        \item Symmetric. (if $(x,y) \in R$, then $(y,x) \in R$). 
        \item Transitive. (if $(x,y) \in R$ and $(y,z) \in R$, then $(x,z) \in R$). 
    \end{enumerate}
    \item \textbf{Equivalence Class. } Let $X$ be a set and $R$ be an equivalence relation on $X$. Then, an equivalence class of $x \in  X$ is the set $[x]=[x]_R=[x]_{\sim} = \{a \in X \mid x \sim a\}$. 
    \item \textbf{$\mathbb{Z}/m\mathbb{Z}. $} The set of distinct equivalence classes of $\equiv \Mod{n}$ is $\mathbb{Z}/m\mathbb{Z}$.
	\item \textbf{Group. } A group $G$ (denote: $(G,\star)$) is a set $G$ with a closed binary operation $\star: G \times G \to G$ such that: 
	\begin{enumerate}
		\item Associativity: $(a \star b) \star c = a \star (b \star c)$ for all $a,b,c \in G$. 
		\item Identity: There exists an $e \in G$ such that for any $a \in G$, we have $a \star e = e \star a = a$. 
		\item Inverse: For any $a \in G$, there exists an $a^{-1} \in G$ such that $a \star a^{-1} = a^{-1} \star a = e$. 
	\end{enumerate}
	\item \textbf{Symmetric Group. } The symmetric group on $n$ letters is $S_n$. 
	\item \textbf{Disjoint Cycles. } Two cycles $(a_1,\dots,a_k)$ and $(b_1,\dots,b_l)$ are disjoint if $a_i \neq b_j$ for all $i,j$. 
	\item \textbf{Transpositions. } The simplest permutation is a cycle of length 2, which is called a transposition. 
	\item \textbf{Even, Odd Permuatations. } A permutation is even if it can be expressed as an even number of transpositions. A permutation is odd if it can be expressed as an odd number of transpositions. 
	\item \textbf{Subgroup. } A subgroup $H$ of a group $G$ is a subset $H$ of $G$ such that when the group operation of $G$ is restricted to $H$, then $H$ is a group. 
	\item \textbf{Trivial/Proper Subgroup. } The trivial subgroup of a group $G$ is $\{e\}$ and a proper subgroup is a subgroup $H$ of $G$ where $H$ is a proper subset of $G$. 
	\item \textbf{General/Special Linear Group. } $GL_2(\mathbb{R})$ is the set of 2x2 invertible matrices with real entries. $SL_2(\mathbb{R})$ is the set of 2x2 invertible matrices with real entries and with determinant 1.
	\item \textbf{Cyclic Group. } A cyclic group is a group generated by one element. 
	\item \textbf{Isomorphism. } An isomorphism is a homomorphism which is bijective. 
	\item \textbf{Kernel of homomorphism. } If $\phi: G \to H$ is a homomorphism, then $\ker \phi$ is the pre-image of $e_H \in H$, that is, $\ker \phi \{g \in G \mid \phi(g)=e_H\}$.
	\item \textbf{Coset. } Let $(G,\star) \geq (H,\star)$ and $g \in G$. Then, an $H-$ coset of $g$ is a (sub)set of $G$ where $gH = g \star H = \{g \star h \mid h \in H\}$ (left coset) and $Hg = \{h \star g \mid h \in H\}$ (right coset). 
	\item \textbf{Index. } A set of distinct equivalence classes with respect to $~_H$ is $G/H$, a quotient of $G$ by $H$. Then, $|G/H| = [G:H]$ is the index of $H$ in $G$. 
	\item The following are definitions listed in the homeworks: 
	\begin{enumerate}
		\item Group of units in $\mathbb{Z}/n\mathbb{Z}$ is the set $(\mathbb{Z}/n\mathbb{Z})^\times = \mathbb{Z}/n\mathbb{Z}^\times := \{[a] \in \mathbb{Z} \mid \exists [b] \in \mathbb{Z}/n\mathbb{Z} \textrm{ with } [a] \times [b] = [1]\}$. 
		\item If $G$ is a group, then the center of $G$ is the set $Z(G) := \{a \in G \mid ga=ag \forall g \in G\}$. 
		\item $\mathbb{C}^\times$ is the set of nonzero complex numbers. 
		\item $\mathbb{R}^\times$ is the set of nonzero real numbers. 
		\item $GL(n,K)$ is the set of $n$ x $n$ invertible matrices with entries in $K$. 
		\item If $G$ is a group, then the torsion subgroup of $G$ is called $G_T$, which is the set of all elements of $G$ with finite order.
		\item The Klein four-group is $V$ is a subgroup of $S_4$ and consists of $V=\{\id, (12), (34), (12)(34)\}$. 
	\end{enumerate} 
	\begin{center}
		\hrule
	\end{center}
	\item \textbf{(External) Direct Product} Let $G = (G,\star)$ and $H = (H, \circ)$ be groups. Then, $G \times H = \{G \times H, (\star, \circ)\}$. 
	\item \textbf{Normal Subgroup. } Let $G$ be a group and $H$ a subgroup of $G$. Then $H$ is a normal subgroup (write $H \unlhd G$) iff for all $g \in G$, $gH=Hg$, or equivalently, for all $h in H$, $ghg^{-1} \in H$ for all $g \in G$. 
\end{enumerate}


\end{document}
