\input{~/normal-preamble.tex}

\begin{center}
    Math 113 Theorems. 
\end{center}

\begin{enumerate}
    \item \textbf{Prop. } The relation $\equiv \Mod{n}$ is an equivalence relation. 
    \item \textbf{Prop. } $\mathbb{Z}/n\mathbb{Z}$ has exactly $n$ elements. 
    \begin{enumerate}
        \item \textbf{Prop 0. } If $i \in [j]$, then $j \in [i]$ (in $\mathbb{Z}/n\mathbb{Z}$). 
        \item \textbf{Prop 1. } If $[i] \cap [j] \neq \emptyset$, then $[i]=[j]$. 
        \item \textbf{Prop 2. } If $i \neq j$ and $0 \leq i,j \leq n-1$, then $[i]\cap[j] = \emptyset$. 
        \item \textbf{Prop 3. } Every $x \in \mathbb{Z}$ belongs to one of $[0],\dots,[n-1]$. 
    \end{enumerate}
	\item \textbf{Prop. } Addition is correctly (well-defined) defined on $\mathbb{Z}/n\mathbb{Z}$ by $[a] + [b] = [a+b]$.
	\item \textbf{Prop 3.17. } The identity element in any group is unique. 
	\item \textbf{Prop 3.18. } The inverse is unique for any element $g$ in a group $G$. 
	\item \textbf{Prop 3.19. } For any $a,b \in G$, where $G$ is a group, $(a \star b)^{-1} = b^{-1}a^{-1}$. 
	\item \textbf{Prop 3.20. } For any $g \in G$, where $G$ is a group, then $(g^{-1})^{-1} = g$. 
	\item \textbf{Theorem 5.1. } $S_n$ is a group with $n!$ elements where the binary operation is the composition of maps. 
	\item \textbf{Prop 5.8. } Let $\sigma$ and $\tau$ be two disjoint cycles in $S_X$. Then, $\sigma\tau = \tau\sigma$. 
	\item \textbf{Theorem 5.9. } Every permutation in $S_n$ can be written as the product of disjoint cycles. 
	\item \textbf{Prop 5.12. } Any permutation of a finite set containing at least 2 elements can be written as the product of transpositions. 
	\item \textbf{Lemma 5.14. } If the identity is written as the product of $r$ transpositions, $\id = \tau_1\dots\tau_r$, then $r$ is even. 
	\item \textbf{Theorem 5.15. } If a permutation $\sigma$ can be expressed as the product of an even number of transpositions, then any other product of transpositions equaling $\sigma$ must also contain an even number of transpositions. Similarly, in the case of when $\sigma$ is odd. 
	\item \textbf{Prop 3.30. } A subset $H$ of $G$ is a subgroup iff: 
	\begin{enumerate}
		\item $e \in G$ also satisfies $e \in H$. 
		\item If $h_1,h_2 \in H$, then $h_1h_2 \in H$. 
		\item If $h \in H$, then $h^{-1} \in H$. 
	\end{enumerate}
	\item \textbf{Prop 3.31. } Let $H$ be a subset of a group $G$. Then, $H$ is a subgroup of $G$ iff $H \neq \emptyset$ and if $g,h \in H$, then $gh^{-1} \in H$. 
	\item \textbf{Theorem 4.3. } Take a group $G$ and an element $a \in G$. Consider a cyclic subgroup $\langle a \rangle$. Then, $\langle a \rangle$ is a minimal subgroup of $G$ such that $a$ is in it (minimality: if $H$ is a subgroup of $G$ and $a \in H$, then $\langle a \rangle$ is a subgroup of $H$). 
	\item \textbf{Theorem 4.9. } Every cyclic group is abelian. 
	\item \textbf{Prop 11.4. } Let $\phi: G \to H$ be a homomorphism. Then: 
	\begin{enumerate}
		\item $\phi(e_G) = e_H$. 
		\item $\phi(g^{-1}) = (\phi(g))^{-1)}$ for all $g \in G$. 
		\item If $K \leq G$, then $\phi(K) := \{\phi(k) \mid k \in K\}$ is a subgroup of $H$. 
		\item $\phi(G) := \{\phi(g) \mid g \in G\}$ (the image of $\phi$) is a subgroup of $H$. 
		\item If $M \leq H$, then $\phi^{-1}(M) := \{g \in G \mid \phi(g) \in M\}$ is a subgroup of $G$. 
	\end{enumerate}
	\item \textbf{Lemma 6.3. } Let $G$ be a group and $H$, a subgroup. Let $g_1,g_2 \in G$. Then, the following are equivalent: 
	\begin{enumerate}
		\item $g_1H = g_2H$. 
		\item $H{g_1}^{-1} = H{g_2}^{-1}$. 
		\item $g_1H \subseteq g_2H$. 
		\item $g_2 \in g_1H$. 
		\item ${g_1}^{-1}g_2 \in H$. 
	\end{enumerate} 
	\item \textbf{Theorem 6.4. } Left $H$-cosets partition $G$. 
	\item \textbf{Lagrange's Theorem. } If $G$ is a finite group and $H$ is a subgroup of $G$, then $|G| = |H| \cdot [G:H]$, or $[G:H] = \frac{|G|}{|H|}$. 
	\item \textbf{Cor. } If $G$ is a finite group and $H$ is a subgroup of $G$, then $|H|$ divides $|G|$. 
	\item \textbf{Cor. 6.13. } If $G$ is a finite group and $H \leq G$ and $G \geq H \geq K$, then $[G:K] = [G:H] \cdot [H:K]$.
	\begin{center}
		\hrule
	\end{center}
	\item \textbf{Prop. } $(\langle (123 \dots n) \rangle, \circ)$ is isomorphic to $(\mathbb{Z}/n\mathbb{Z},+)$. 
	\item \textbf{Theorem 9.7. and 9.8} If $G = (G, \star)$ is cyclic, then if: 
	\begin{enumerate}
		\item $G$ finite, then $G$ is isomorphic to $(\mathbb{Z}/n\mathbb{Z}, +)$. 
		\item $G$ infinite, then $G$ is isomorphic to $(\mathbb{Z},+)$. 
	\end{enumerate}
	\item \textbf{Prop. } Assume $G$ is abelian. Then every subgroup of $G$ is normal. 
	\item \textbf{Prop. } Take $G = \mathbb{Z}$, $H = n\mathbb{Z}$, $a,b \in \mathbb{Z}$. Then $aH \odot bH$ gives $(a+n\mathbb{Z}) \odot (b+n\mathbb{Z}) = (a + b) + n\mathbb{Z}$ is correctly defined. 
	\item \textbf{Theorem. } Let $G$ be a group and $H$ a normal subgroup. Then $\odot$ (as in the above Prop.) defines a group structure on $G/H$, where $G/H$ is called a quotient (factor) group. 
	\item \textbf{Prop. } Let $\phi: G \to K$ be a homomorphism. Then, $\ker \phi$ is a normal subgroup of $G$, with $\ker\phi \unlhd G$. 
	\item \textbf{First Isomorphism Theorem. } Let $\phi: G \to H$ be a homomorphism. Then $G/\ker\phi \cong \textrm{Im}\phi$ and denote $\Phi: G/\ker\phi \to \textrm{Im}\phi$ with $g \cdot \ker\phi \mapsto \phi(g)$. 
	\item \textbf{Theorem 9.27. } If $G$ is an internal direct product of $H$ and $K$ (with $H,K \leq G$), then, $G \cong H \times K$, where $G$ represents an internal direct product and $H \times K$ represents an external direct product. 
	\item \textbf{Fundamental Theorem of Finite Abelian Groups. } Every finite abelian group $G$ is isomorphic to one of the following form: $G \cong \mathbb{Z}/p_1^{a_1}\mathbb{Z} \times \dots \times \mathbb{Z}/p_m^{a_m}\mathbb{Z}$ for $p_1,\dots,p_m$ primes and $a_1,\dots,a_m \in \mathbb{Z}_{>0}$, where $|G| = p_1^{a_1} \cdot \cdot \cdot p_m^{a_m}$. 
	\item \textbf{Cor. } Any abelian group with 6 elements is isomorphic to $\mathbb{Z}/2\mathbb{Z} \times \mathbb{Z}/3\mathbb{Z}$. 
	\item \textbf{Prop. } If $G$ is a finite group with $p$ elements (where $p$ is prime), then $G \cong \mathbb{Z}/p\mathbb{Z}$. 
	\item \textbf{Prop. } If $|G| = 4$, then $G$ is abelian. 
	\item \textbf{Prop. } If for any $a \in G$, $a^2 = e_G$, then $G$ is abelian. 
	\item \textbf{Prop. } $\textrm{Sym(cube)} \cong S_4$, so there are 24 symmetries of the cube, looking at the symmetry of the set of all 4 long diagonals inside the cube. 
	\item \textbf{Prop. } Let $G$ be a group and $X$ a set. Then, for each $x \in X$, we have $\Stab_G(x) \leq G$. 
	\item \textbf{Prop. } If $G$ acts on a set $X$ and both $G$ and $X$ are finite, then $|G|=|\Stab_G(x)| \cdot |\orb(x)|$ for all $x \in X$. 
	\item \textbf{Prop. } If $G$ acts on $X$, then $G$ acts by bijection, i.e. $\{x \mid x \in X\} = \{g \circ x \mid x \in X\}$ (in bijection for any $g \in G$). 
	\item \textbf{Prop. } For any sets $A,B$ (that contain identity), with $A \xrightarrow[]{\psi} B$ and $A \xleftarrow[]{\phi} B$ with $\phi \circ \psi = \id_A$ and $\psi \circ \phi = \id_B$, then both $\phi$ and $\psi$ are bijections. 
	\item \textbf{Prop. } The two definitions of actions are equivalent, i.e. $\{\Phi: G \times X \to X\}$ (with properties 1 and 2 as in the (equivalent) definition of $G$ acting on $X$) is equal to the set $\{\phi: G \to \Sym(X)\}$, where $\phi$ is a homomorphism. 
	\item \textbf{Cayley's Theorem. } Every group is isomorphic to a subgroup of $S_n$. 
	\item \textbf{Lemma. } Let $\lambda: G \to \Sym(G)$ with be the left regular action of a group $G$ on $G$. Then, $\lambda$ is injective. 
	\item \textbf{Burnside's Lemma. } Let $G$ be a finite group with $G$ acting on a finite set $X$. The number of $G$-orbits in $X$ is $\frac{1}{|G|} \cdot \sum_{g \in G} |X^g|$, where $|X^g|$ is the number of elements in $X$ fixed by the action of $g \in G$. 
	\item \textbf{Theorem. } The set of normal subgroups in $G$ is equal to the set of all $\ker \phi$ where $\phi: G \to H$ is a homomorphism. 
	\item \textbf{Prop. } $\ker\phi$ is an ideal in $R$ for any ring homomorphism $\phi: R \to S$. 
	\item BELOW ARE ADDITIONAL THMS FOR MT2
	\item \textbf{Prop 16.8. } Let $R$ be a ring with $a,b \in R$. Then: 
	\begin{enumerate}
		\item $a0=0a=0$. 
		\item $a(-b)=(-a)b=-ab$. 
		\item $(-a)(-b) = ab$. 
	\end{enumerate}
	\item \textbf{Prop 16.10. } Let $R$ be a ring and $S$ a subset of $R$. Then $S$ is a subring of $R$ iff: 
	\begin{enumerate}
		\item $S \neq \emptyset$. 
		\item $rs \in S$ for all $r,s \in S$. 
		\item $r-s \in S$ for all $r,s \in S$. 
	\end{enumerate}
	\item \textbf{Prop. 16.15. Cancellation Law. } Let $D$ be a commutative ring with identity. Then $D$ is an integral domain iff for all nonzero elements $a \in D$ with $ab=ac$, we have $b=c$. 
        \item \textbf{Theorem 16.16. } Every finite integral domain is a field. 
        \item \textbf{Lemma 16.18. } Let $R$ be a ring with identity. If 1 has order $n$, then the characteristic of $R$ is $n$. 
        \item \textbf{Theorem 16.19. } The characteristic of an integral domain is either prime or zero. 
	\item \textbf{Prop. 16.22. } Let $\phi:R \to S$ be a ring homomorphism. Then: 
        \begin{enumerate}
            \item If $R$ is a commutative ring, then $\phi(R)$ is a commutative ring. 
            \item $\phi(0)=0$. 
            \item Let $1_R$ and $1_S$ be the identities for $R$ and $S$, respectively. If $\phi$ is onto, then $\phi(1_R) = 1_S$. 
            \item If $R$ is a field and $\phi(R) \neq \{0\}$, then $\phi(R)$ is a field. 
        \end{enumerate}
        \item \textbf{Theorem 16.25. } Every ideal in the ring of integers $\mathbb{Z}$ is a principal ideal. 
        \item \textbf{Prop. 16.27. } The kernel of any ring homomorphism $\phi: R \to S$ is an ideal in $R$. 
	\item \textbf{First Ring Isomorphism Theorem. } Let $\psi: R \to S$ be a ring homomorphism. Then $\ker \psi$ is an ideal of $R$. if $\phi: R \to R/\ker\psi$ is the canonical homomorphism, then there exists a unique homomorphism $\nu: R/\ker\psi \to \psi(R)$ such that $\psi = \nu\phi$. 
	\begin{center}
		\hrule
	\end{center}
\end{enumerate}

\end{document}
