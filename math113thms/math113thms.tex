\documentclass[12pt]{article}
% \usepackage[left=2cm, right=2cm, top=1.5cm, bottom=1.5cm]{geometry}
\usepackage{amsmath}
\usepackage{amsthm}
\usepackage{amsfonts}
\usepackage{amssymb}
\usepackage{authblk}
\usepackage{tkz-euclide}
\usepackage{tikz}
\usepackage{changepage}
\usepackage{lipsum}
\usepackage{tree-dvips}
\usepackage{qtree}
\usepackage[linguistics]{forest}
\usepackage[hidelinks]{hyperref}
\usepackage{mathtools}
\usepackage{blindtext}
% \usepackage[cal=esstix,frak=euler,scr=boondox,bb= pazo]{mathalfa}
% the following 2 packages are used for changing the font. 
\usepackage{mathptmx}
\usepackage{mathrsfs}
\usepackage{graphicx}
\usepackage{setspace}
\graphicspath{{./images/}}
\allowdisplaybreaks
\allowbreak
\theoremstyle{definition}
\newtheorem{definition}{Definition}
\newtheoremstyle{named}{}{}{\itshape}{}{\bfseries}{.}{.5em}{\thmnote{#3's }#1}
\theoremstyle{named}
\newtheorem*{namedconjecture}{Distinct Factorizations Conjecture}
\newtheorem{conjecture}{Conjecture}
\DeclareMathOperator{\sech}{sech}
\DeclareMathOperator{\arcsec}{arcsec}
\DeclareMathOperator{\lcm}{lcm}
\DeclareMathOperator{\curl}{curl}
\DeclareMathOperator{\Res}{Res}
\DeclareMathOperator{\Aut}{Aut}
\DeclareMathOperator{\id}{id}
\DeclareMathOperator{\nul}{nul}
\newcounter{customDef}
\renewcommand{\thecustomDef}{\arabic{customDef}}
\newcommand{\Mod}[1]{\ (\mathrm{mod}\ #1)}
\begin{document}

\begin{center}
    Math 113 Theorems. 
\end{center}

\begin{enumerate}
    \item \textbf{Prop. } The relation $\equiv \Mod{n}$ is an equivalence relation. 
    \item \textbf{Prop. } $\mathbb{Z}/n\mathbb{Z}$ has exactly $n$ elements. 
    \begin{enumerate}
        \item \textbf{Prop 0. } If $i \in [j]$, then $j \in [i]$ (in $\mathbb{Z}/n\mathbb{Z}$). 
        \item \textbf{Prop 1. } If $[i] \cap [j] \neq \emptyset$, then $[i]=[j]$. 
        \item \textbf{Prop 2. } If $i \neq j$ and $0 \leq i,j \leq n-1$, then $[i]\cup[j] = \emptyset$. 
        \item \textbf{Prop 3. } Every $x \in \mathbb{Z}$ belongs to one of $[0],\dots,[n-1]$. 
    \end{enumerate}
	\item \textbf{Prop. } Addition is correctly (well-defined) defined on $\mathbb{Z}/n\mathbb{Z}$ by $[a] + [b] = [a+b]$.
	\item \textbf{Prop 3.17. } The identity element in any group is unique. 
	\item \textbf{Prop 3.18. } The inverse is unique for any element $g$ in a group $G$. 
	\item \textbf{Prop 3.19. } For any $a,b \in G$, where $G$ is a group, $(a \star b)^{-1} = b^{-1}a^{-1}$. 
	\item \textbf{Prop 3.20. } For any $g \in G$, where $G$ is a group, then $(g^{-1})^{-1} = g$. 
	\item \textbf{Theorem 5.1. } $S_n$ is a group with $n!$ elements where the binary operation is the composition of maps. 
	\item \textbf{Prop 5.8. } Let $\sigma$ and $\tau$ be two disjoint cycles in $S_X$. Then, $\sigma\tau = \tau\sigma$. 
	\item \textbf{Theorem 5.9. } Every permutation in $S_n$ can be written as the product of disjoint cycles. 
	\item \textbf{Prop 5.12. } Any permutation of a finite set containing at least 2 elements can be written as the product of transpositions. 
	\item \textbf{Lemma 5.14. } If the identity is written as the product of $r$ transpositions, $\id = \tau_1\dots\tau_r$, then $r$ is even. 
	\item \textbf{Theorem 5.15. } If a permutation $\sigma$ can be expressed as the product of an even number of transpositions, then any other product of transpositions equaling $\sigma$ must also contain an even number of transpositions. Similarly, in the case of when $\sigma$ is odd. 
	\item \textbf{Prop 3.30. } A subset $H$ of $G$ is a subgroup iff: 
	\begin{enumerate}
		\item $e \in G$ also satisfies $e \in H$. 
		\item If $h_1,h_2 \in H$, then $h_1h_2 \in H$. 
		\item If $h \in H$, then $h^{-1} \in H$. 
	\end{enumerate}
	\item \textbf{Prop 3.31. } Let $H$ be a subset of a group $G$. Then, $H$ is a subgroup of $G$ iff $H \neq \emptyset$ and if $g,h \in H$, then $gh^{-1} \in H$. 
	\item \textbf{Theorem 4.3. } Take a group $G$ and an element $a \in G$. Consider a cyclic subgroup $\langle a \rangle$. Then, $\langle a \rangle$ is a minimal subgroup of $G$ such that $a$ is in it (minimality: if $H$ is a subgroup of $G$ and $a \in H$, then $\langle a \rangle$ is a subgroup of $H$). 
	\item \textbf{Theorem 4.9. } Every cyclic group is abelian.  
\end{enumerate}

\end{document}
