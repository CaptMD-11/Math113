\input{~/preamble.tex}

\begin{center}
	TEST
\end{center}
DEFS
\begin{enumerate}
	\item A set is an unordered collection of elements. 
	\item A map from a set $X$ to a set $Y$ is a rule (writ: $f: X \to Y$) such that for each $x \in X$ there exists a unique $y \in Y$ such that $f(x)=y$. 
	\item Let $X$ and $Y$ be sets. then their cartesian prdouct is $X \times Y = \{(x,y) \mid x \in X, y \in Y\}$. 
	\item Let $X$ be a set. then an equivalence relation is $R \subseteq X \times X$ with:
	\begin{enumerate}
		\item reflexive: $x \sim x$ for all $x \in R$. 
		\item symmetric: $x \sim y$ means $y \sim x$ for all $x,y \in R$. 
		\item transitive: $x \sim y$ and $y \sim z$ means $x \sim z$ for all $x,y,z \in R$. 
	\end{enumerate}
	\item Let $R$ be an equivalence relation with $R \subseteq X \times X$. Then, the equivalence class of $x \in R$ is $[x] = \{a \in R \mid x \sim a\}$. 
	\item $\mathbb{Z}/m\mathbb{Z}$ is the set of equivalence classes of the integers mod $m$. 
	\item A group $G$ is a set $G$ equipped with a binary operation $\circ$ such that: 
	\begin{enumerate}
		\item if $a,b \in G$, then $a \circ b \in G$. 
		\item $(a \circ b) \circ c = a \circ (b \circ c)$ for all $a,b,c \in G$. 
		\item there exists $e \in G$ such that $a = a \circ e = e \circ a$ for all $a \in G$. 
		\item for each $a \in G$, there exists $a' \in G$ such that $a \circ a' = a' \circ a = e \in G$. 
	\end{enumerate}
	\item Symmetric group is the group of permutations on $n$ letters, write $S_n$. 
	\item let $(a_1,\dots,a_k)$ and $(b_1,\dots,b_m)$ be cycles. Then, they are disjoint cycles if $a_i \neq b_j$ for all $i,j$. 
	\item a transposition is a cycle of length 2, the simplest possible permutation. 
	\item a permutation is called odd if it can be written as the product of an odd number of transpositions, similarly for even permuattions. 
	\item if $G$ is a group, then we call $H$ a subgroup of $G$ if $H$ is a subset of $G$ and $H$ is a group under the same binary operation as on $G$. 
	\item let $G$ be a group. Then the trivial subgroup is $\{e\}$ and $H$ is a proper subgroup of $G$ if $H$ is a proper subset of $H$ and $H$ is a subgroup of $G$. 
	\item The general linear group $GL_2(\mathbb{R})$ is the set of 2x2 invertible matrices of real entries and the special linear group $SL_2(\mathbb{R})$ is the set of 2x2 matrices of real entries and determinant 1. (both are groups under the binary operation of multiplication of their respective elements). 
	\item A cyclic group is a group such that the entire group is genereated by a single element. 
	\item an isomorphism is a homomorphism that is bijective (that is, 1-1 and onto). 
	\item Let $f: G \to H$ be homomorphism. then $\textrm{ker} f$ is the set $\{g \in G \mid f(g)=e_H\}$. 
	\item Let $G$ be a group and $H$, a subgroup. then the left $H$-coset of $g \in G$ is the set $gH = \{gh \mid h \in H\}$ and the right $H$-coset of $g \in G$ is the set $Hg = \{hg \mid h \in H\}$. If the right and left $H$-cosets are indistinguishable, then we call them both just cosets. 
	\item Let $G$ be a group and $H$, a subgroup. we define then $G/H$ to be the quotient group, that is the group of all equivalence classes with respect to $H$ in $G$. then, we say the index of $H$ in $G$ is $|G/H| = [G:H]$. 
\end{enumerate}

THMS
\begin{enumerate}
	\item the relation $\equiv \Mod{n}$ is an equivalence relation on $\mathbb{Z}$. 
	\item $\mathbb{Z}/n\mathbb{Z}$ has exactly $n$ elements. 
	\begin{enumerate}
		\item if $i \in [j]$, then $j \in [i]$ (in $\mathbb{Z}/n\mathbb{Z}$). 
		\item if $[i] \cap [j] \neq \emptyset$, then $[i]=[j]$. 
		\item if $i \neq j$ and $0 \leq i < j \leq n-1$, then $[i] \cap [j]=\emptyset$. 
		\item every $x \in Z$ belongs to one of $[0],\dots,[n-1]$. 
	\end{enumerate}
	\item Addition is correctly and well-defined on $\mathbb{Z}/n\mathbb{Z}$ as $[a]+[b] = [a+b]$. 
	\item the identity element in a group is unique. 
	\item the inverse for an element in a group is unique.
	\item for any $a,b \in G$, $(ab)' = b'a'$
	\item for any $g \in G$, $g'' = g$. 
	\item $S_n$ is a group with $n!$ elements with the binary operation being composition of permutations. 
	\item Let $\sigma, \tau$ be disjoint cycles in $S_X$. then $\sigma\tau = \tau\sigma$. 
	\item every permutation in $S_n$ can be written as the product of disjoint cycles. 
	\item any permutation of a finite set of at least 2 elements can be written as the product of transpositions. 
	\item if the identity $\id$ is written as the product of $r$ transpositions, then $r$ is even. 
	\item if a permutation $\sigma$ can be expressed as the product of an even number of transpositions, then any product of transpositions equaling $\sigma$ must contain an even number of cycles. similarly, for odd. 
	\item $H$ is a subgroup of a group $G$ iff: 
	\begin{enumerate}
		\item if $a \in H$, then $a' \in H$. 
		\item if $a,b \in H$, then $ab \in H$. 
		\item the identity of $G$ exists in $H$, and is $H$'s identity element. 
	\end{enumerate}
	\item let $H$ be a subset of a group $G$. then $H$ is a subgroup iff $H \neq \emptyset$ and $gh' \in H$ for all $g,h \in H$. 
	\item take a group $G$ and $a \in G$. consider the cyclic subgroup $\langle a \rangle$. then $\langle a \rangle$ is a minimal subgroup of $G$ that contains $a$, where minimality means that if $H$ is a subgroup of $G$ and $a \in H$, then $\langle a \rangle$ is a subgroup of $H$. 
	\item every cyclic group is abelian. 
	\item let $\phi: G \to H$ be a homomorphism. Then: 
	\begin{enumerate}
		\item $\phi(e_G) = e_H$. 
		\item $\phi(g)' = \phi(g')$ for all $g \in G$. 
		\item let $K$ be a subgroup of $G$. then $\phi(K) = \{phi(g) \mid g \in K\}$ is a subgroup of $H$. 
		\item $\phi(G) \leq H$, where $\phi(G) = \{\phi(g) \mid g \in G\}$. 
		\item let $M$ be a subgroup of $H$. then $\phi'(M) = \{g \in G \mid \phi(g) \in M\}$ is a subgroup of $G$. 
	\end{enumerate}
	\item TFAE: (let $G$ be a group, and $H$ a subgroup of $G$) and $g_1,g_2 \in G$. 
	\begin{enumerate}
		\item $g_1H = g_2H$. 
		\item $Hg_1' = Hg_2'$. 
		\item $g_1 \in g_2H$. 
		\item $g_1H \subseteq g_2H$. 
		\item $g_1'g_2 \in H$. 
	\end{enumerate}
	\item left $H$-cosets partition $G$. 
	\item lagrange's theorem. let $G$ be a finite group and $H$ a subgroup of $G$. then $[G:H] = \frac{|G|}{|H|}$, or $|G| = |H| \cdot [G:H]$. 
	\item cor. let $G$ be a finite group and $H$ a subgroup of $G$. then $|H|$ divides $|G|$ and $[G:H]$ divides $|G|$. 
	\item let $G$ be a finite group and $G \geq H \geq K$. Then $[G:K] = [G:H] \cdot [H:K]$. 
\end{enumerate}

\end{document}
