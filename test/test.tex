\input{~/dense-preamble.tex}

\begin{center}
	TEST
\end{center}
DEFS
\begin{enumerate}
	\item a set is an uniorderd colection oof item. 
	\item a map ffrom a set $X$ to a set $Y$ is $f: X \to Y$ wherre eleemtsn in $Y$ are assigned to elements of $X$, where for each $x \in X$, htere is a unique $y \in Y$ such that $f(x)  = y$. 
	\item a cartesian proudct between sets $X$ and $Y$ is $X \times Y = \{(x,y) \mid x \in X, y \in Y\}$. 
	\item an equivlanece relation $R.\tilde$ on a set $X$ is a subset $R \subseteq X \times X$ such that it is refliexive $x \sim x$ for all $x \in R$, $x \sim y$ means $y \sim x$ for all $x,y \in R$, and transitive, $x \sim y \land y \sim z$ imples $x \sim z$ for all $x,y,z \in R$. 
	\item let $\sim$ be an equivflacce relation on a set $X$ then the equivalcenc class of $x \in X$ is the set $\{a \in X \mid x \sim a\}$. 
	\item $Z_n$ is the set of equivlane classes for the relation $\equiv$ $(mod n)$. 
	\item a grup $G$ is a a set with an binary opeation defined on it such that the opeation is closed and $G$ contains identity, closed under taking inverses, and is associative. 
	\item symmetric group on $n$ letters is $S_n$, which is also the group of permutations on $n$ elements. 
	\item disjoint cycles are $(a_1,\dots,a_n)$ and $(b_1,\dots,b_m)$ where $a_i \neq b_j$ for all $i,j$. 
	\item transposition is the simplest type of permuation , a cycle of length 2. 
	\item an even permutation is a permutation that can be written as a product of an even number of transpositions, and similary for eodd. 
	\item a subgroup $H$ of $G$ is a susbet $H$ of $G$ such that $H$ is a group on its own when the binary operation of $G$ is restricted to $H$. 
	\item trivial group is the $\{e\}$ and a proper subgroup of $G$ is $H$, where $H$ is a subgroup of $G$ where $H$ is also a proper subset of $G$. 
	\item $GL_2(R)$ is the group of 2x2 inveritble matrices with real entries (where the group operation is matrix multiplication) and $SL_2(R)$ is the group of 2x2 matrics with real entries and determinant 1. 
	\item a cyclic group is  group with only one generator. 
	\item isomoorphism is a bijective homomorphism. 
	\item kernel of homomorphism is the preimage of the identity in the codomain. 
	\item let $G$ be a group and $H$ a subgroup then a left coset of $H$ is $gH$ and simiarly for right coset. 
	\item The set of distinct equivalence classes of is $G/H$ and the number of such equivlanece classes (number of left cosets of $H$ in $G$) is the index of $H$ in $G$. 
	\item dihedral group is the group of all symmetris of a regular $n$-gon, call it $D_n$, where we have that it is generated by $r,s$ and $r$ is rotation by $\2pi / n$ and $s$ is a flip about an axis of reflectoin. We have that $r^n = \id = s^2$ and $srs = r^{-1}$. 
	\item external direct product of groups $G,H = G \times H = \{(g,h) \mid g \in G, h \in H\}$, where we have that $(a,b) \star (c,d) = (ac,bd)$, where have respective group operations defined (3 of them). 
	\item Let $G$ be a group and $H$ a subgroup. $H$ is a normal subgroup of $G$ iff $gH = Hg$ for all $g \in G$, and for an $h \in H$, $ghg^{-1}$ is in $H$ for all $g \in G$. 
	\item let $G$ be a group and $N$ a normal subgroup. then the quotient subgroup is $G/N$, which the set of all left cosets of $N$ in $G$. 
	\item let $G$ be a group and $H,K \leq G$. Then $G$ is the internal direct prodcut of $H$ and $K$ iff: 
	\begin{enumerate}
		\item $HK = \{hk \mid h \in H, k \in K\}$. 
		\item $H \cap K = \{e\}$. 
		\item $hk = kh$ for all $k \in K$, $h \in H$. 
	\end{enumerate}
	\item a simple group is group whose only normal subgroups are itself and the trvial subgroup. 	
	\item a symmetry of $X$ is a bijective map $\sigma: X \to \Sym(X)$, that preserves the structure, where $X$ has some additional structure. 
	\item the group of permutations on a set $X$. Let $G$ be a group. $G$ is a group of permutations on a set $X$ if we have the homomorphism $\phi: G \to \Sym(X)$ that is 1-1. 
	\item $G$ acts on a set $X$ is a homomorphism $\phi: G \to \Sym(X)$. 
	\item stabililizer of $x \in X$ is the set $\{g \in G \mid gx = x\}$ where a agruop $G$ acts on a set $X$. 
	\item orbit of $x \in X$ is the set $\{gx \mid g \in G\}$. 
	\item $G$ acts on a set equivalent defintion is the map $\Phi: G \times X \to X$, where we have that $Phi(g,x) = g \circ x $, and 
	\begin{enumerate}
		\item $\Phi(e,x)=x$ for all $x \in X$. 
		\item $\Phi(gh,x) = \Phi(g,\Phi(h,x))$. 
	\end{enumerate}
	\item let $G$ be a group and let it act on itself. Then we have that $\Phi: G \times G \to G$ is the group acting on itself. Then the left regular action of $G$ on itself is the homomorphism $\phi: G \to \Sym(G)$, where we have $g \mapsto (\lambda_g: h \mapsto g \circ h)$, where $\lambda_g$ is a permutatsion on the set $G$. 
	\item a ring is a set $R$ with two closed binary operations such that: 
	\begin{enumerate}
		\item $(R,+)$ is an abelian group. 
		\item $(R,\times)$ is associative. 
		\item both distributive properteis hold. 
	\end{enumerate}
	\item $R^\times$ is the set of all elemetns of ring $R$ that have multiplicative inverse. 
	\item a field is a commutative ring $R$ such that each nonzero element of it has a multiplicative invers. 
	\item ring homomorhsim is $\phi: R \to S$ such that $\phi(a+b) = \phi(a) + \phi(b)$ and $\phi(ab) = \phi(a)\phi(b)$. 
	\item 
\end{enumerate}
\end{document}
