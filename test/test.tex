\input{~/dense-preamble.tex}

\begin{center}
	TEST
\end{center}
DEFS
\begin{enumerate}
	\item a proper ideal of a ring is an ideal $I \leq R$ so that $I$ is not zero ideal and I is not $R$. 
	\item an integral domain is a commutatitve ring with idneity so that $R$ has no nonzero zero divisors. 
	\item an ideal $I$ of a ring $R$ is a prime ideal iff $ab \in I$ implies $a \in I$ or $b \in I$. 
	\item let $D$ be an integral domain. $p \in D$ is prime iff $p \mid ab \implies p \mid a \lor p \mid b$. 
	\item let $D$ be an ID. then $p \in D$ irredubible iff $p = ab \implies$ a or b is a unit. 
	\item a principal ideal domain is an integral domain $D$ so that every ideal in $D$ is a principal ideal. 	
	\item a unique factorization domain (UFD) is an integral domain so that (1): let $a \in D$ nonzero, nonunit. then we have that a can be written as a product of irreducibesl in $D$. (2): if $a=p_1,\dots p_n = q_1 \dots q_m$, then $n=m$ and there exists $\pi \in S_n$ so that $p_i, q_{\pi(j)}$ are associates. 
	\item An integral domain $D$ is a euclidean domain with a fucntion $\nu: D \setminus \{0\} \to \mathbb{N}$ so that (1): if $a,b \in D$ nonzero, then $\nu(a) \leq \nu(ab)$. (2): if $a,b \in D$ and $b \neq 0$, then there exist $q,r \in D$ so that $a=bq+r$ with $r = 0$ or $\nu(r) < \nu(b)$. 
	\item let $I,J$ be ideals. Then their product is $I \cdot J = \{\sum_{k=1}^n a_kb_k \mid a_k \in I, b_k \in J, n \in \mathbb{Z}_{>0}\}$. 
	\item let $R$ be a ring. it is finitely generrated iff there exist $x_1,\dots,x_n \in R$ (for some $n \in \mathbb{Z}_{>0}$) so that $R = \langle x_1,\dots,x_n \rangle = \{a_1x_1 + \dots + a_nx_n \mid a_i \in R \forall i\}$. 
	\item let $R$ be a commutative ring with identity. then $x,y \in R$ are asssociates iff there exist $c \in U(R)$ so that $x=yc$. 
	\item a vector space is a set $V$ so that 
	\begin{enumerate}
		\item $(V,+)$ is an abelian group, with so that $+: V \times V \to V$ so that $(u,v) \mapsto u+v$. 
		\item $V$ closed under scalar multiplication, i.e. if $\lambda \in F$ and $v \in V$, then $\lambda \cdot v \in V$. 
		\item $(\alpha\beta)v = \alpha(\beta v)$. 
		\item $\alpha(u+v) = \alpha u + \alpha v$. 
		\item $(\alpha + \beta)v = \alpha v + \beta v$. 
		\item $1 \cdot v = v$. 
	\end{enumerate}
	\item a linear map from $F$-vector spaces $V,W$ is a map $\phi: V \to W$ so that $\phi(v+w) = \phi(v) + \phi(w)$ and $\phi(cv) = c\phi(v)$. 
	\item let $E \geq F$ be fields with field extension. then $E \geq F$ is a simple algebraic field extension iff $E = F(\alpha_1)$ where $\alpha \in E$ and algebraic over $F$. 
	\item let $E$ and $F$ be fields. $E \geq F$ is a simple algebraic field extension iff $E \cong \sfrac{F[x]}{\langle p(x) \rangle}$ for irreducible $p(x)$. 
	\item $F[p(x)] := F(\alpha_1,\dots,\alpha_n)$, where $\alpha_1,\dots,\alpha_n$ are all the roots of $p(x)$. 
	\item let $F$ be a field and $p(x)$ be a nonconstant polynomial in $F[x]$. Then $E \geq F$ is a splitting field of the extension $E \geq F$ iff $E = F(\alpha_1,\dots,\alpha_n)$ for some $\alpha_i \in E (\forall i)$ and so that $p(x) = \prod (x-\alpha_i)$. 
	\item the degree of a splitting field extension is the dimension of the vector space it genrates. 
	\item let $E \geq F$ be a field extension. then an automorphism of $E$ over $F$ is a bijective ring homoomrphism $\phi: E \to E$ so that $\phi(f) = f \forall f \in F$. 
\end{enumerate}
	
\begin{enumerate}
	\item if $T$ is a field, then its only ideals are the zero one and $T$ itself. 
	\item $R/I$ is a field iff $I$ is maximal in $R$. 
	\item for a given ring, the set of zero divisors is disjoint from the set of units. 
	\item $R/I$ is an integral domain iff $I$ is a prime ideal in $R$. 
	\item div alg. Let $a,b \in Z$ so that $b \neq 0$. Then there exist unique $q,r \in Z$ so that $a=bq+r$ with $0 \leq r < b$. 
	\item let $a,b \in Z$ nonzero. then there exist integers $p,q \in Z$ so that $\gcd(a,b) = pa+qb$, where $\gcd(a,b)$ is unique. 
	\item fundamental theorem of arithemetic. let $n \in \mathbb{Z}_{>0}$. then $n$ can be factored uniquely into product of primes. 
	\item if $a(x),b(x) \in F[x]$, then there exist unique $q(x),r(x) \in F[x]$ so that 
	\begin{enumerate}
		\item $a(x) = q(x)b(x)+r(x$. 
		\item $\deg(r(x)) < \deg(b(x))$. 
	\end{enumerate}
	\item if $F$ is a field, then $\alpha \in F$ so that $\alpha$ is a root of $p(x) \in F[x]$ iff $(x - \alpha) \mid p(x)$. 
	\item if $F$ is any field and $f(x) \in F[x]$ has degree $n$, then $f(x)$ has at most $n$ roots. 
	\item $Z[i]$ is a commutative ring with identity but not a field. 
	\item units in $Z[i]$ are $\pm 1$ and is an integral domain. 
	\item $N(xy) = N(x)N(y)$ for all $x,y$ in the gaussian integers. 
	\item div. alg. for guasssian integers. let $\alpha,\beta \in Z[i]$, and also let $\beta \neq 0$. then there exist $q,r \in Z[i]$ so that $\alpha = q\beta + r$ where $r = 0$ or $N(r) < N(b)$. 
	\item $Z[sqrt -5]$ is an integral domain, commutative ring with 1, but not a field. 
	\item the units in z sqrt -5 are $\pm 1$ and is an integral domain. 
	\item norm is multiplicative on z sqrt -5
	\item let $R$ be an integral domain. then every prime is irreducible. 
	\item $3$ is irreducible in z sqrt -5 but not prime
	\item let $R$ be a integral domain. then $\langle u \rangle = R$ iff $u$ is a unit in $R$. 
	\item let $R$ be an integral and $r \in R$ a nonunit. then $\langle r \rangle$ is prime iff $r$ is prime. 
	\item if $I = <a>$ and $J=<b>$, then $I \cdot J = <ab>$, where $I$ and $J$ are (principal) ideals in $R$. 
	\item if $R = Z$ and $x_1,\dots,x_n \in Z$ then $\langle x_1,\dots,x_n \rangle = \{a_1x_1 + \dots + a_nx_n, \mid a_i \in R \forall i\} = <\gcd(x_is>$. 
	\item z sqrt -5 is not a principal ideal domain. 
	\item let $R$ be z sqrt -5. for norm we have that $N(A) = 0$ iff $A = 0$, $\alpha$ is a unit iff $N(\alpha) = \pm 1$, and if $N(\alpha)$ is prime, then $\alpha$ is irreducible. 
	\item every maximal ideal of a commutativie ring with identiy is a prime ideal .
	\item fundamental theorem of ideal theory. let $R$ be a commutative ring with identity. and $I$ be a nonzero proper ideal of $R$. then there exists a unique (up to order) factorization $I = P_1 \cdot \cdot \cdot P_k$ (for some $k$) so that each $P_i$ is a prime ideal. 
	\item let $\alpha$ be a nonzero nonunit element of z sqrt -5. then $\alpha$ is irreducible iff $<\alpha>$ is prime, or iff $<\alpha> = P_1 \cdot P_2$, where $P_i$ are nonprincipal prime ideals. 	
	\item if $\alpha$ is a nonzero elemtn of z sqrt -5, and $\beta_1,\dots,\beta_n$ and $\gamma_1,\dots,\gamma_m$ are irreducbible, with the products both equal to $\alpha$, then $s=t$. 
	\item if $F$ is a field then $F[x]$ is a PID. 
	\item let $p(x) \in F[x]$. then $\langle p(x) \rangle$ is maximal iff $p(x)$ is irreducible. 
	\item in $F[x]$, prime ideals = maximal ideals. 
	\item if $E \geq F$ is a field extension, then $E$ is an $F$-vector spcae. 
	\item splitting field algorithm. let $F$ be a field and $p(x) \in F[x]$ irred. then to find the splitting field of $F[p(x)]$, we have that put $F_1 := \sfrac{F[x]}{<p(x)>}$, and write $p(x) = (x-a_1)q(x)$, put $F_2$ and so on.... 
	\item let $F=K(a_1,a_2)$ be a field extension. then the degree $[F:K] = [F:K(a_1)] \cdot [K(a_1):K]$. 
	\item $\{\id,\conj\}$ are all the automorphisms of $C$ over $R$. 
\end{enumerate}

\end{document}
