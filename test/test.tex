\input{~/dense-preamble.tex}

\begin{center}
	TEST
\end{center}
DEFS
\begin{enumerate}
	\item a set is an unordered collection of elements. 
	\item a map from a set $X$ to a set $Y$ (write: $f: X \to Y$) is a rule that assigns elements of $Y$ to elements of $X$, that is, for each $x \in X$ there exists a unique $y \in Y$ such that $f(x)=y$. 
	\item Let $X$ and $Y$ be sets. then their cartesian product is $X \times Y = \{(x,y) \mid x \in X, y \in Y\}$. 
	\item let $X$ be a set and consider a relation $R \subseteq X \times X$. Then $R$ is an equivalence relation if: 
	\begin{enumerate}
		\item $x \sim x$ for all $x \in R$. 
		\item $x \sim y$ means $y \sim x$ for all $x,y \in R$. 
		\item $x \sim y$ and $y \sim z$ means $x \sim z$ for all $x,y,z \in R$. 
	\end{enumerate}
	\item let $R$ be an equivalence relation on $X$. then the equivalence class of $x \in R$ is $[x] = \{a \in X \mid x \sim a\}$. 
	\item $\mathbb{Z}/n\mathbb{Z}$ is the set of equivalence classes of integers mod $n$ (the relation being $\equiv_n$). 
	\item a group $G$ is a set $G$ with a binary operation $\circ$ such that the following hold: 
	\begin{enumerate}
		\item if $a,b in G$, then $a \circ b \in G$. 
		\item $(a \circ b) \circ = a \circ (b \circ c)$ for all $a,b,c \in G$. 
		\item there exists $e \in G$ such that $a \circ e = e \circ a = a$ for all $a \in G$.
		\item for each $a \in G$ there exists $a' \in G$ such that $a \circ a' = a' \circ a = e \in G$. 
	\end{enumerate}
	\item the symmetric group $S_n$ is the group of all permutations of $n$ letters, and the binary operation on $S_n$ is the operation of composition of permutations on the $n$ letters. 
	\item two cycles $(a_1,\dots,a_k)$ and $(b_1,\dots,b_l)$ are disjoint if $a_i \neq b_j$ for all $i,j$. 
	\item a cycle of length 2 is called a transposition (the simplest type of permutation). 
	\item a permutation is even if can be expressed as a product of an even number of transpositions, and similarly for odd permutations. 
	\item Let $G$ be a group. $H$ is a subgroup of $G$ if $H$ is a subset of $G$ and $H$ is a group under the same binary operation as defined on $G$. 
	\item let $G$ be a group. The trivial subgroup of $G$ is just $\{e\}$. a proper subgroup of $G$ is a subgroup of $G$ that is also a proper subset of $G$. 
	\item the general linear group is $GL_2(\mathbb{R})$, which is the set of 2x2 invertible matrices of real entries. the special linear group is $SL_2(\mathbb{R})$, which is the set of 2x2 matrices of real entries and determinant 1. 
	\item a cyclic group is a group that is generated by one of its elements. 
	\item an isomorphism is a homomorphism that is bijective. 
	\item let $\phi: G \to H$ be a homomorphism. then the kernel of $\phi$ is the set $\textrm{ker}\phi = \{g \in G \mid \phi(g) = e_H\}$. 
	\item let $G$ be a group and $H$ a subgroup. then the left $H$-coset of $g \in G$ is the set $gH = \{gh \mid h \in H\}$. the right $H$-coset of $g \in G$ is $Hg = \{hg \mid h \in H\}$. if the left and right $H$-cosets of $g \in G$ are indistinguishable, then we just call them both cosets. 
	\item let $G$ be a group and $H$ a subgroup. then we define $G/H$ to be the set of equivalence classes with respect to $H$ in $G$. then we say $[G:H] = |G/H|$ is the index of $H$ in $G$. 
\end{enumerate}

THMS
\begin{enumerate}
	\item the relation $\equiv_n$ is an equivalence relation on $\mathbb{Z}$. 
	\item $\mathbb{Z}/n\mathbb{Z}$ has exactly $n$ elements. 
	\begin{enumerate}
		\item if $i \in [j]$ then $j \in [i]$ (in $\mathbb{Z}/n\mathbb{Z}$). 
		\item if $[i] \cap [j] \neq \emptyset$, then $[i]=[j]$. 
		\item if $i \neq j$ and $0 \leq i < j \leq n-1$ then $[i] \cap [j] = \emptyset$. 
		\item each $x \in \mathbb{Z}$ lies in exactly one of $[0],\dots,[n-1]$. 
	\end{enumerate}
	\item addition is correctly \& and well-defined on $\mathbb{Z}/n\mathbb{Z}$ to be $[a] + [b] = [a+b]$. 
	\item the identity element in a group $G$ is unique. 
	\item if $G$ is a group, the inverse of $g \in G$ is unique. 
	\item for all $a,b$ in a group $G$, $(ab)' = b'a'$. 
	\item for all $g in G$ where $G$ is a group, then $g'' = g$. 
	\item let $\sigma$ and $\tau$ be disjoint cycles on $S_X$ then $\sigma\tau = \tau\sigma$. 
	\item every permutation in $S_n$ can be written as the product of disjoint cycles. 
	\item any permutation of a finite set consisting of at least 2 elements can be written as the product of transpositions. 
	\item if the identity $\id$ is written as the product of $r$ transpositions, then $r$ is even. 
	\item if a permutation $\sigma$ can be written as the product of an even number of transpositions, then any product of transpositions equaling $\sigma$ must contain an even number of transpositions. Similarly for odd. 
	\item a subset $H$ is a subgroup of a group $G$ iff: 
	\begin{enumerate}
		\item $e_G \in G$ is also the identity element in $H$. 
		\item if $a,b$ in $H$, then $ab \in H$. 
		\item for each $a \in H$, $a' \in H$. 
	\end{enumerate}
	\item let $H$ be a subset of a group $G$. then $H$ is a subgroup of $G$ iff $H \neq \emptyset$ and $gh' \in H$ for all $g,h \in H$. 
	\item take a group $G$ and an element $a \in G$. then the cyclic subgroup $\langle a \rangle = \{a^k \mid k \in \mathbb{Z}\}$ is a minimal subgroup of $G$ containing $a$. minimality is that if $H$ is a subgroup of $G$ and $a \in H$, then $\langle a \rangle$ is a subgroup of $H$. 
	\item every cyclic group is abelian. 
	\item let $\phi: G \to H$ be a homomorphism. Then: 
	\begin{enumerate}
		\item $\phi(e_G) = e_H$. 
		\item $\phi(g') = \phi(g)'$ for all $g \in G$. 
		\item if $K$ is a subgroup of $G$, then $\phi(K) = \{\phi(g) \mid g \in K\}$ is a subgroup of $H$. 
		\item $\phi(G) = \{\phi(g) \mid g \in G\}$ is a subgroup of $H$. 
		\item if $M$ is a subgroup of $H$, then its pre-image $\phi'(M) = \{g \in G \mid \phi(g) \in M\}$ is a subgroup of $G$. 
	\end{enumerate}
	\item let $G$ be a group and $H \leq G$. also let $g_1,g_2 \in G$. then, TFAE: 
	\begin{enumerate}
		\item $g_1H = g_2H$. 
		\item $Hg_1' = Hg_2'$. 
		\item $g_2 \in g_1H$. 
		\item $g_1H \subseteq g_2H$. 
		\item $g_2'g_1 \in H$. 
	\end{enumerate}
	\item left $H$-cosets partition $G$. 
	\item lagrange's theorem. let $G$ be a finite group and $H$ a subgroup of $G$. then $[G:H] = \frac{|G|}{|H|}$, or $|G| = [G:H] \cdot |H|$. 
	\item cor. let $G$ be a finite group and $H$ a subgroup. then $|H|$ and $[G:H]$ divide $|G|$. 
	\item cor. let $G$ be a finite group and $H,K$ subgroups with $K \leq H \leq G$. then $[G:K] = [G:H] \cdot [H:K]$. 
\end{enumerate}

\end{document}
